\section{학술 대회 논문 작성시 유의 사항}

\subsection{논문 페이지 수}
\begin{itemize}[itemsep=0pt,parsep=0pt]
  \item 2쪽 이상 4쪽 이내 
\end{itemize}


\subsection{용지 및 여백 처리}
\begin{itemize}[itemsep=0pt,parsep=0pt]
  \item 용지 : A4, 가로쓰기
  \item 여백 : 위 쪽 30mm, 아래 쪽 20mm, 왼 쪽 10mm, 오른 쪽 10mm
\end{itemize}

\subsection{논문 구성}
아래 순서대로 작성하며, ①∼⑧항목은 1 단 ⑨∼⑪항목은 2 단으로 구성


\begin{enumerate}[itemsep=0pt,parsep=0pt]
	\renewcommand{\theenumi}{\arabic{enumi}}
	\renewcommand{\labelenumi}{\cl\theenumi}
	\item 제목(국문)
	\item 저자명(국문) * 발표자는 공동저자와 구분 처리 \\ (예) 홍길동$^{\circ}$
	\item 소속(국문)
	\item 저자 E-mail Address
	\item 제목(영문)
	\item 저자명(영문) * 발표자는 공동저자와 구분처리 \\ (예) Kildong Hong$^{\circ}$
	\item 소속(영문)
	\item 요약
	\item 본문
	\begin{itemize}[itemsep=0pt,parsep=0pt]
	  \item 장 및 절에 해당되는 번호는  아라비아 숫자로 각각 1., 1.1 등과 같이 표기
	  \item 그림의 명칭은 하단에, 표는 상단에 그림 1 및 표 1로 표기
	\end{itemize} 
	\item 참고 문헌
	\begin{itemize}[itemsep=0pt,parsep=0pt]
	  \item 본문중에 \cite{Lee:2008,Myung:2008,Lee:2010}과 같이 참고문헌 번호를 쓰고, 그 문헌을 참고 문헌란에 인용한 순서대로 기술
	  \item 기술 순서는 저자, 제목, 학술지명, 권, 호, 쪽수, 발행년도 순으로 작성.
	\end{itemize}
	\item 부록(해당사항이 있는 경우만 작성) 
\end{enumerate}



\subsection{기타}
\begin{itemize}[itemsep=0pt,parsep=0pt]
  \item 위 유의사항 3개항목을 제외한 논문작성폰트, 크기는 임의 사용가능합니다. 
  단, 논문집(Proceedings) 제작시 축소 인쇄하므로 글자크기를 9pt 이하는 사용하지 마시기 바랍니다.
  \item 논문심사는 저자와 심사위원 상호 비공개로 진행됩니다. 
  따라서, 심사용(저자정보 삭제)과 출판용(저자정보 포함)으로 나눠 제출합니다. 
  심사용은 투고시, 출판용은 심사후 지정된 수정기간중에 각 업로드 하시면 됩니다. 
\end{itemize}

\section{\LaTeX 사용 시 유용한 팁}

\LaTeX을 사용하여 논문을 작성하면 다양한 이점이 있다.
먼저 복잡한 수식을 쉽게 작성할 수 있다.
가장 큰 장점 중 하나는 bibtex을 이용해서 참고 문헌 관리를 편리하게 할 수 있다는 것이다.
이 외에도 그림이나 표의 참조를 쉽게 할 수 있다는 것 등 다양한 장점이 있다.

여기서는 \LaTeX을 이용하여 논문을 작성할 때 많이 사용되는 명령에 대한 몇 가지 예를 제시한다.
자세한 내용은 \LaTeX에 관한 다양한 메뉴얼을 참고하기 바란다.

\subsection{문서 서식 사용}
본 latex 템플릿을 사용하는 경우에는 기본적인 여백, 폰트 크기가 KCC에서 요구하는 기본 서식을 준수한다.
또한, 문서 서식(documentclass)을 지정할 때 옵션으로 {\tt preprint}를 주면 저자 정보가 제외된다.


\subsection{수식}
\begin{equation}
\label{eq:eq1}
SumOnlyPositives(\mathcal{D}) = \sum_{\forall x \in \mathcal{D} \land x > 0 }x
\end{equation}
\eqref{eq:eq1}과 같이 수식을 바로 참조할 수 있다.
\index{equation}


\subsection{알고리즘}
Algorithm \ref{alg:sum}과 같이 알고리즘을 표현할 수 있다.
\begin{algorithm}[!ht]
\label{alg:sum}
\caption[Short label]{\texttt{SumOnlyPositives}($\mathcal{D}$)}
\begin{algorithmic}[1]
\REQUIRE a set of real number $\mathcal{D}$ 
\ENSURE sum of all non-negative elements in $\mathcal{D}$ 
\STATE $O \leftarrow 0$
\FOR{\textbf{each} $x\in \mathcal{D}$ }
	\IF{$x > 0$}  
		\STATE $O \leftarrow O + x$
	\ENDIF
\ENDFOR
\RETURN $O$
\end{algorithmic}
\end{algorithm}

  
\subsection{표}
표 \ref{tab:datasets}과 같이 표를 작성할 수 있다.
\begin{table}[!ht]
\centering
\setlength{\belowcaptionskip}{5pt}
\caption{실험 데이터의 주요 통계적 수치}
\label{tab:datasets}
\begin{tabular}{@{}lrrrr@{}} 
\toprule
{\bfseries Dataset} & $|\mathcal{D}|$ & ${\rm avg}(|x|)$ & $|\mathcal{U}|$ & ${\rm avg}(|I_i|)$ \\
\midrule
LAST.FM			&134,949	&  4.8	& 47,295	& 13.8\\
LAST.FM 4G		&			& 11.2	& 44,272	& 34.3\\
DBLP			&1,298,016	&  8.6	&381,450	& 29.3\\
TREC			&348,566	& 77.1	&298,302	& 90.1\\
UKBENCH			&10,200		&425.7	&533,412	&  6.9\\
\bottomrule
\end{tabular}
\end{table}


\subsection{그림}

그림 \ref{fig:example1}과 같이 외부 그림을 삽입할 수 있다.

\begin{figure}[!ht]
\centering
\includegraphics[width=0.4\textwidth]{figures/features.pdf}
\caption{그림 예제}
\label{fig:example1}
\end{figure}
\index{figure}


\subsection{도형 직접 그리기}

그림 \ref{fig:picture}와 같이 몇 가지 명령을 이용해서 간단한 그림을 직접 그릴 수 있다.

\begin{figure}[h!]
\centering
\setlength{\unitlength}{6pt}
\begin{picture}(40,10)
\put(20,5){\circle{6}$y$}
\put(3,2){\framebox(5,4){$x$}}
\end{picture}
\caption{간단한 도형 그리기 예제}
\label{fig:picture}
\end{figure}



\subsection{참고 문헌 관리}
각각의 참고 문헌을 bibtex 형식으로 관리하고,
다양한 형식으로 쉽게 출력할 수 있다.
bibtex에 관한 다양한 참고 문헌을 참조하기 바란다.
컴퓨터 공학 관련 분야에서는 {\em ieeetr, IEEE, unsrt, plain, abbrv}와 같은 형식이 자주 쓰인다.